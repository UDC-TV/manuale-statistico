\section*{\LARGE{APPENDICE}}
\sectionline

\begin{enumerate}
    \Large{} % il più classico dei "NON RIMUOVERE"
    \item \Large{TIPI DI TIRO}
    \item \Large{TIPI DI PALLE PERSE}
    \item \Large{DATI AGGIUNTIVI}
    \item \Large{ISTANT REPLAY SYSTEM (IRS) E HEAD COACH CHALLENGE (HCC)}
\end{enumerate}

\newpage

\setcounter{section}{0}
% Cambia numerazione: sezioni con lettere
% Non includere queste sezioni nell'indice
\renewcommand{\thesection}{\Alph{section}}
\section{TIPI DI TIRO}
\sectionline

Se il software richiede l'inserimento del tipo di tiro, sono disponibili i seguenti tipi, o un loro sottoinsieme.

Al fine di mantenere i tipi di tiro gestibili per gli statistici meno esperti, si raccomanda di consentire al software di operare con set di tiro di base e completi, in base al livello della competizione.

\subsection*{TIPI DI TIRO BASE}
\subsectionline
\minititle{TIRO IN SOSPENSIONE}
\qbox{\textbf{Jumpshot}. Tiro effettuato saltando in aria e rilasciando la palla al culmine del salto. Si usa soprattutto per i tiri dalla media e lunga distanza, compresi i tentativi da tre punti.}{}

\minititle{LAY-UP}
\qbox{iro da distanza ravvicinata, in cui il tiratore di solito spinge la palla contro il tabellone; tuttavia, questo tiro può essere eseguito anche oltre il bordo del campo. In genere si tratta di un tiro a una mano eseguito tenendo la palla dal basso e rilasciandola con un movimento del braccio verso l'alto, il più vicino possibile al canestro. Il lay-up può anche essere un tiro rapido in cui l'attaccante prende la palla vicino al canestro.}{}

\minititle{SCHIACCIATA}
\qbox{La schiacciata è un'azione in cui un giocatore porta la palla al di sopra dell'anello e cerca di schiacciarla a canestro con una o due mani, e la mano o le mani colpiscono la parte superiore dell'anello.}{}

\minititle{SCHIACCIATA PUTBACK}
\qbox{Quando un giocatore della squadra in attacco prende un rimbalzo offensivo e poi sbatte immediatamente la palla con forza a canestro, con una o due mani, e le sue mani colpiscono la parte superiore dell'anello.}{}

\minititle{TIP-IN PUTBACK}
\qbox{Quando un giocatore della squadra offensiva prende un rimbalzo offensivo e poi segna immediatamente vicino al canestro. Spesso questo comporta che il giocatore offensivo salti in alto e rovesci la palla nel canestro con una mano, senza tornare a terra.}{}

\subsection*{TIPI DI TIRO AVANZATI}
\subsectionline
\minititle{LAY-UP CON PALLEGGIO}
\qbox{Tiro da distanza ravvicinata dopo un'azione di palleggio verso il canestro, quando la difesa è arretrata o in contropiede. Di solito il tiro è appoggiato al tabellone, ma può anche essere eseguito sopra il bordo del campo.}{}

\minititle{LAY-UP ROVESCIATO}
\qbox{Azione in cui l'attaccante guida da un lato del canestro e realizza un lay-up dall'altro lato, utilizzando il canestro/il tabellone come ulteriore protezione contro la stoppata di un difensore.}{}

\minititle{EURO STEP}
\qbox{L'euro step è un movimento in cui il giocatore offensivo compie due passi, di solito lunghi, dopo un palleggio, dove il primo passo è in una direzione e il secondo nell'altra.}{}

\minititle{ALLEY-OOP}  
\qbox{Tiro che coinvolge un giocatore che riceve un passaggio in aria e termina con una schiacciata, prima di atterrare nuovamente in campo.}{}  

\minititle{GANCIO}  
\qbox{Tiro a una mano in cui il giocatore con la palla si gira lateralmente verso il canestro con la mano di tiro più lontana dal canestro, quindi estende il braccio di tiro e fa scorrere la palla sopra la testa con un movimento circolare verso il canestro.}{}  

\minititle{TIRO FLUTTUANTE}  
\qbox{\textbf{Floater.} La palla viene tirata da un piede senza fermarsi, spesso come un driving lay-up ma più lontano dal canestro. Il tiro può essere effettuato in corsa, in palleggio o in presa, può essere effettuato in appoggio o in tiro, e spesso ha un arco alto che non può essere bloccato.}{}  

\minititle{TIRO IN FADE-AWAY}  
\qbox{Quando il tiro viene effettuato mentre il giocatore sta saltando lontano dal canestro. Questo tiro viene utilizzato per creare spazio tra il tiratore e il suo difensore, può essere tentato da qualsiasi punto del campo e può essere effettuato con un salto all'indietro o di lato.}{}  

\minititle{TIRO CON GIRO}  
\qbox{Quando il giocatore riceve la palla con le spalle al canestro e inizia il tiro rivolto verso il canestro prima di girarsi a mezz'aria mentre salta per tirare. Il giocatore può girarsi completamente verso il canestro, ma a volte si gira solo parzialmente e tira con la faccia rivolta verso il canestro.}{}  

\minititle{TIRO IN STEP-BACK}  
\qbox{Il giocatore di solito finge di guidare verso il canestro, poi si ferma e fa un passo indietro per creare spazio tra sé e il difensore prima di tirare.}{}  

\minititle{TIRO IN PULL-UP}  
\qbox{Il tiratore si ferma rapidamente dal palleggio e si alza per tirare un tiro in sospensione, mentre i difensori sono di solito ancora in posizione bassa a difendere il drive.}{}

\newpage
\subsection*{TIPI DI PALLE PERSE}
L'elenco seguente definisce i valori possibili per i tipi di palla persa.:

\begin{itemize}
    \item Passaggio sbagliato
    \item Gestione della palla (ball-handling)
    \item Palla fuori dal terreno di gioco (OOB)
    \item Passi
    \item 3 secondi
    \item 5 secondi
    \item 8 secondi
    \item Violazione dei 24 secondi
    \item Ritorno della palla in zona di difesa (backcourt violation) (RPZD)
    \item Fallo offensivo
    \item Fallo da espulsione della squadra in possesso del pallone
    \item Interferenza a canestro offensiva
    \item Doppio palleggio
    \item Palla accompagnata
    \item Altro
\end{itemize}

\newpage
\section{DATI AGGIUNTIVI}
\sectionline

In questo allegato vengono definiti alcuni dati aggiuntivi che sono tipicamente calcolati dal software e quindi non direttamente rilevanti per il lavoro di uno statistico.

\subsection*{TEMPO/MINUTI GIOCATI}
\subsectionline
Tutte le sostituzioni vengono inserite nel software e per ogni giocatore viene calcolato il tempo di gioco.

Se i minuti giocati sono indicati solo in minuti (cioè senza secondi), si applica il seguente sistema di arrotondamento:
\begin{itemize}
    \item I minuti con meno di 30 secondi saranno arrotondati per difetto.
    \item I minuti con 30 secondi o più saranno arrotondati per eccesso.
    \item 0 minuti saranno arrotondati per eccesso a 1 minuto, indipendentemente dal valore dei secondi (maggiore di 0).
    \item Qualsiasi valore con 1 minuto in meno rispetto al tempo massimo (ad esempio, 39 minuti per una partita giocata 4x10 minuti) sarà arrotondato per difetto per qualsiasi valore di secondi, per indicare che il giocatore non ha giocato l'intera partita.
\end{itemize}

Un giocatore che non è entrato in campo viene indicato con "DNP" (did not play), invece di un valore per minuti e secondi. 

Qualsiasi tempo di gioco di un giocatore compreso tra 0,1 secondi e 0,9 secondi sarà arrotondato al secondo intero.

Ai fini statistici, una partita con DNP non conta come una partita giocata per il giocatore.

\subsection*{PUNTI NEL PITTURATO}
\subsectionline
Il numero totale di punti segnati da una squadra da un FGM che ha origine all'interno dell'area riservata. Sono compresi tutti i tiri in sospensione, i tiri in gancio, i lay-up, le schiacciate, ecc.

\subsection*{PUNTI DA FUORIGIOCO}
\subsectionline
Il numero totale di punti segnati da una squadra durante il possesso successivo a un palla persa avversario. Questo vale indipendentemente dal tipo di palla persa, dal fatto che la palla esca dal campo e che i punti provengano da un FGM o da un FTM.

Questo non si applica a un FGM o FTM in un possesso aggiuntivo, a seguito di un fallo chiamato contro un giocatore della squadra in difesa dopo un FGA o FGM.

\subsection*{PUNTI DI CONTROPIEDE SU PALLA PERSA}
\subsectionline
Il numero totale di punti in contropiede segnati da una squadra durante il possesso successivo a un palla persa avversario. Questo vale indipendentemente dal tipo di palla persa, dal fatto che la palla esca dal campo e che i punti provengano da un FGM o da un FTM.

Questo non si applica a un FGM o FTM in un possesso supplementare, a seguito di un fallo chiamato contro un giocatore della squadra in difesa dopo un FGA o FGM.

\subsection*{PUNTI DA SECONDA OPPORTUNITÀ}
\subsectionline
Il numero totale di punti segnati da una squadra dopo un rimbalzo offensivo e prima che gli avversari ne riprendano il possesso. Questo vale indipendentemente dal fatto che la palla esca dal campo e i punti possono derivare da un FGM o da un FTM.

Questo non si applica a un FGM o FTM in un possesso supplementare, a seguito di un fallo chiamato contro un giocatore della squadra in difesa dopo un FGA o FGM.

\subsection*{PUNTI IN PANCHINA}
\subsectionline
Il numero totale di punti segnati da una squadra escludendo i cinque giocatori titolari.

\subsection*{PUNTEGGIO PAREGGIATO}
\subsectionline
Il numero di volte in cui il punteggio è stato in parità durante la partita (escluso lo 0-0).

\subsection*{VANTAGGIO CAMBIATO}
\subsectionline
Il numero di volte in cui il vantaggio è passato da una squadra all'altra durante la partita.

\subsection*{VANTAGGIO PIÙ GRANDE}
\subsectionline
Il vantaggio maggiore di ciascuna squadra durante la partita e il momento in cui si è verificato (periodo e ora).

\subsection*{MAGGIOR NUMERO DI PUNTI SEGNATI}
\subsectionline
Il maggior margine di punti consecutivi segnati da una squadra senza che gli avversari ne segnino alcuno.

\subsection*{RATING DI EFFICIENZA}
\subsectionline
L'indice di efficienza è spesso indicato per ogni giocatore nei tabellini. Esistono diverse formule per questo dato, ma quella raccomandata come standard è: 

PTS - (FGA-FGM) - (FTA-FTM) + REB + AST - TO + ST + BS

\subsection*{STATISTICHE +/-}
\subsectionline
Le statistiche +/- di ogni giocatore sono spesso riportate nei tabellini. Indica i punti netti della squadra nel periodo in cui il giocatore era in campo.

\subsection*{PUNTI PER POSSESSO}
\subsectionline
Numero di punti segnati da una squadra diviso per il numero di possessi di quella squadra.

\newpage
\section{ISTANT REPLAY SYSTEM (IRS) E HEAD COACH CHALLENGE (HCC)}
\sectionline

Gli statistici devono registrare qualsiasi uso di IRS e HCC, compreso il tipo di revisione. Sono disponibili i seguenti tipi di revisione:

\begin{itemize}
    \item Fine periodo (tiro rilasciato prima del segnale acustico)
    \item Correzione del cronometro di gara/24 secondi
    \item Violazione di 24 secondi (tiro rilasciato prima del segnale acustico)
    \item Fallo lontano dalla situazione di tiro (il canestro vale/non vale)
    \item Interferenza a canestro (il canestro vale/non vale)
    \item Palla fuori campo (OOB)
    \item Valore del FGM
    \item 2 o 3 FT assegnati per un fallo su tiro
    \item Elevazione (\textit{upgrade}) o diminuzione (\textit{downgrade}) di fallo(i)
    \item Identificare il tiratore di tiro libero
    \item Atto di violenza
\end{itemize}