\section{FALLI}
\sectionline
\subsection{DEFINIZIONE}
\subsectionline

Un fallo viene chiamato contro un giocatore in seguito a una decisione dell'arbitro. I falli personali, di rimessa, tecnici, antisportivi e da espulsione possono essere fischiati contro un giocatore. I falli tecnici e da espulsione possono essere fischiati contro un allenatore o il personale della panchina. È importante differenziare i tipi di fallo, se il software delle statistiche lo consente. I falli tecnici e da espulsione chiamati contro l'allenatore o il personale della panchina 
sono registrati contro l'allenatore e non sono conteggiati come falli di squadra.

Un fallo di tiro è qualsiasi fallo chiamato dagli arbitri come fallo "nell'atto di tirare". Un fallo che determina FT solo a causa di falli di squadra è un fallo 
non di tiro.

I falli contro entrambe le squadre in cui le penalità sono annullate a causa della parità di penalità saranno registrati come falli con 0 tiri liberi. Falli commessi \textcolor{violet}{???}

Ogni volta che un giocatore subisce un fallo, gli viene accvioletitato un fallo tirato. Nel caso di un fallo da espulsione, se il fallo è fisicamente commesso su un giocatore, un fallo tirato viene accvioletitato al giocatore che ha subito il fallo.

\subsection{ESEMPI}
\subsectionline
\minititle{ESEMPIO 1}
\qbox{A3 sta palleggiando la palla e commette una carica su B2.}{palla persa (fallo offensivo) A3 - Fallo personale A3 - Fallo tirato B2 (nessuna rubata per B2).}

\minititle{ESEMPIO 2}
\qbox{A2 sta tenendo la palla quando subisce un fallo 
da B2.}{Fallo di A2 - Fallo personale di B2.}

\minititle{ESEMPIO 3}
\qbox{A1 commette un fallo da espulsione
\\a. insultando l'arbitro o
\\b. colpendo B2 con il gomito.}
{a. Fallo da espulsione A1 - palla persa A1;
\\b. Fallo da espulsione A1 - palla persa A1 - Fallo tirato B2.}
