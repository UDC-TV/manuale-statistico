\section{STOPPATE}
\sectionline
\subsection{DEFINIZIONE}
\subsectionline

una stoppata viene assegnato a un giocatore ogni volta che entra in contatto in modo significativo con la palla per alterare il volo di una FGA e il tiro 
viene mancato. una stoppata è un chiaro arresto o deviazione di un tiro da parte di un giocatore difensivo. Una stoppata può essere registrato 
indipendentemente dal fatto che la palla abbia lasciato o meno la mano dell'attaccante.

L'atto del tiro, ai fini statistici, è un movimento verso l'alto e/o in avanti verso il canestro con l'intenzione di tentare un goal. Nelle situazioni in cui la palla viene fatta cadere prima che sia in volo, il tiro viene registrato come FGA, Tiro bloccato e Rimbalzo:

\begin{itemize}
    \item Se la palla si trova al di sopra dell'altezza delle spalle, vengono registrati un FGA, una stoppata e un rimbalzo.
    \item Se la palla è al di sotto dell'altezza delle spalle, si registrano un palla persa e una rubata se la squadra in difesa ottiene il possesso; la rubata 
    deve essere conteggiata solo se la palla rimane viva; se nella stessa situazione la squadra in attacco rimane in possesso, non si registra 
    alcuna statistica.
\end{itemize}

Come per qualsiasi altro FGA mancato, un rimbalzo deve seguire una stoppata, a meno che il periodo non finisca o si verifichi una violazione dello shot clock immediatamente dopo la stoppata.

\subsection{ESEMPI}
\subsectionline
\minititle{ESEMPIO 1}
\qbox{Il \#6 rosso tira e la palla viene toccata da \#7 bianco nel tentativo di bloccare il tiro. La palla finisce nel canestro.}{Poiché la palla è finita nel canestro, il tocco di \#7 bianco non ne ha alterato sensibilmente il volo. Ignorare il tocco, accreditare a \#6 rosso un FGM, ma non accreditare a \#7 bianco una stoppata.}

\minititle{ESEMPIO 2}
\qbox{Il \#9 bianco sta tentando un lay-up e si vede sottrarre la palla all'altezza della vita dal \#10 blu. La palla viene recuperata da \#0 blu.}{Palla persa (gestione della palla) \#9 bianco - rubata \#10 blu.}

\minititle{ESEMPIO 3}
\qbox{Il \#13 verde tira un FGA, che viene bloccato dal \#5 bianco mentre la palla è in movimento verso l'alto e la palla esce dal campo.}{FGA \#13 verde, stoppata \#5 bianco - Rimbalzo offensivo della squadra verde.}

\minititle{ESEMPIO 4}
\qbox{A2 tenta un tiro da tre punti. La palla rimbalza sul anello e B1 ne impedisce l'entrata ribaltandola. A3 raccoglie la palla vagante.}{FGA A2 - Rimbalzo offensivo A3. Nota: una stoppata non può essere registrato dopo che la palla ha colpito l'anello.}

\minititle{ESEMPIO 5}
\qbox{A1 sta guidando verso il canestro e subisce un fallo da B1 nell'atto di tirare. A1 riesce a rilasciare il tiro, che viene bloccato dal difensore in aiuto B2 (il  tiro sarebbe stato conteggiato se realizzato).}{Fallo personale B1 - Fallo subito A1. In questa situazione non si applicano altre statistiche (poiché il tiratore subisce un fallo, non viene registrata alcuna FGA e senza una FGA non può esserci una stoppata).}
