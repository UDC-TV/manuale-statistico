\section{INTRODUZIONE}
\sectionline

Il ruolo degli statistici è quello di registrare ciò che è accaduto sul campo. Il presente manuale fornisce linee guida rigide e rapide per ridurre la probabilità che gli statistici tirino a indovinare. Per qualsiasi situazione non contemplata nel presente manuale, gli statistici devono usare il loro miglior giudizio per registrare il gioco. Nelle situazioni di esempio, la squadra A si riferisce alla squadra in attacco, mentre la squadra B si riferisce alla squadra in difesa. Le interpretazioni ufficiali sono riportate in ogni sezione. Per svolgere il proprio ruolo, gli statistici devono comprendere a fondo il \href{https://about.fiba.basketball/en/services/resource-hub/downloads}{Regolamento Ufficiale della Pallacanestro}.

\newpage
\setcounter{section}{0}
\renewcommand{\thesection}{\arabic{section}}
\section{POSSESSO}
\sectionline

\subsection{DEFINIZIONE}
\subsectionline

Molte delle seguenti definizioni si riferiscono alla definizione di possesso.
Il possesso di una squadra inizia quando un giocatore di quella squadra ha il controllo di un pallone vivo tenendolo o palleggiandolo, oppure quando un pallone vivo è a disposizione del giocatore stesso

\begin{itemize}

    \item Il possesso inizia quando un giocatore di quella squadra ha il controllo di un pallone vivo tenendolo o palleggiandolo, oppure quando un pallone vivo è a disposizione del giocatore stesso. Il possesso continua quando un giocatore di quella squadra ha il controllo della palla o la passa tra compagni di squadra.
    \item Il possesso termina quando la squadra avversaria / un giocatore della squadra avversaria ne acquisisce il controllo; il possesso può terminare dopo una palla persa, un tiro a canestro realizzato (FGM) / ultimo tiro libero realizzato (FTM) o un tiro a canestro tentato (FGA) / ultimo tiro libero tentato (FTA) (seguito da un rimbalzo difensivo)

\end{itemize}

Un tiro mancato (FGA o ultimo FTA) seguito da un rimbalzo offensivo è considerato come una continuazione dello stesso possesso, non come un nuovo o ulteriore possesso. 

\subsection{MINI-POSSESSI}
\subsectionline

I mini-possessi sono situazioni in cui il controllo della palla cambia per un breve periodo di tempo, da una squadra all'altra, per poi tornare alla squadra originale. Ai fini statistici i mini-possessi non vengono considerati come cambi di possesso.

È importante ricordare che \textbf{la definizione di controllo e quella di possesso in ambito statistico, sono simili ma non identiche a quelle del Regolamento Ufficiale della Pallacanestro}.