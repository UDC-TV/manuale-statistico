\section{RIMBALZI}
\sectionline

\subsection{DEFINIZIONE}
\subsectionline

Ogni FGA o ultimo FTA non realizzato, è seguito da un rimbalzo, ad eccezione dei casi elencati qui sotto. Un rimbalzo è il recupero controllato di una palla viva da parte di un giocatore o una squadra, incluso il diritto alla palla per una rimessa a seguito di un FGA o ultimo FTA sbagliato.

I rimbalzi sono divisi in offensivi e difensivi. Gli offensivi sono registrati quando il possesso è acquisito da un giocatore della squadra che ha sbagliato il FGA o FTA, mentre i difensivi sono registrati quando il possesso è acquisito dalla squadra che non ha cercato di segnare.

\begin{itemize}

    \item Il rimbalzo può essere assegnato con le seguenti azioni:

        \begin{itemize}
            \item Essere il primo a controllare la palla, anche se essa è stata toccata diverse volte, è rimbalzata o è rotolata sul terreno di gioco.
            \item Effettuare un \textit{tip} nel tentativo di realizzare un canestro.
            \item Effettuare un \textit{tip} o deviare la palla in maniera intenzionale e controllata, a un compagno di squadra o in una posizione sul terreno di gioco dove può essere recuperata facilmente da un compagno di squadra.
            \item Effettuare un \textit{tip} o deviare la palla in maniera controllata verso un avversario, facendogli commettere una violazione, con il risultato che il possesso viene assegnato alla squadra del giocatore che ha effettuato il \textit{tip} o la deviazione.
            \item Prendere il rimbalzo contemporaneamente a un avversario, e vedere la palla assegnata alla propria squadra per la rimessa di possesso alternato.
        \end{itemize}

    \item Un rimbalzo di squadra è registrato alla squadra che ha diritto alla palla quando:

        \begin{itemize}
            \item La palla esce dal terreno di gioco (OOB) dopo un FTA o FGA sbagliato e prima che un giocatore acquisisca il controllo.
            \item Viene fischiato un fallo dopo un FGA o FTA , prima che un giocatore acquisisca il controllo della palla.
            \item Dopo un FTA o FGA sbagliato, due o più giocatori della stessa squadra sono coinvolti in una situazione di palla a due.
            \item La palla si appoggia tra anello e tabellone, o sul supporto del canestro.
            \item Viene realizzato un autoncanestro durante una situazione a rimbalzo (e.g., un giocatore effettua un \textit{tip} e la palla entra nel proprio canestro).
        \end{itemize}

    \item Nessun rimbalzo deve essere registrato:

        \begin{itemize}
            \item Dopo un qualsiasi FTA dove la palla non è viva.
            \item Alla fine del quarto, quando il segnale acustico suona dopo un FTA o FGA sbagliato, e prima che una squadra acquisisca il possesso.
            \item Dopo un FGA sbagliato, dove la palla non tocca l'anello, il cronometro dei 24 secondi suona, e gli arbitri fischiano per segnalare una violazione di 24 secondi prima che un giocatore controlli la palla.
        \end{itemize}
\end{itemize}


\subsection{MINI POSSESSI}
\subsectionline
In situazioni dove un un rimbalzo del giocatore è immediatamente seguito da una palla persa di quello stesso giocatore, (e.g., controllare la palla in aria e poi atterrare fuori dal terreno di gioco), un rimbalzo di squadra deve essere assegnato alla squadra avversaria. "Controllo" in questo contesto, è definito come il giocatore che controlla la palla e il proprio corpo.

\subsection{ESEMPI}
\subsectionline

\minititle{ESEMPIO 1}
\qbox{Un tiro sbagliato è recuperato simultaneamente dal \#9 bianco e dal \#27 blu.}{Il rimbalzo sarà registrato al giocatore la cui squadra ha diritto alla palla per la rimessa di possesso alternato (e.g., rimbalzo offensivo \#9 bianco).}

\minititle{ESEMPIO 2}
\qbox{Dopo un tiro sbagliato del \#2 rosso, il \#5 bianco salta e prende la palla, ma cade e perde equilibrio. Il \#5 bianco perde quindi il controllo della palla che viene recuperata dal \#7 rosso.}{Rimbalzo offensivo \#7 rosso.}

\minititle{ESEMPIO 3}
\qbox{Dopo un tiro sbagliato, il \#14 bianco prende la palla e quasi nello stesso momento subisce fallo dal \#24 rosso.}{Lo statistico deve valutare se il \#14 bianco aveva il controllo della palla per una frazione di secondo prima di subire fallo. In caso affermativo, il rimbalzo dovrà essere registrato al giocatore; altrimenti dovrà essere registrato un rimbalzo difensivo di squadra.}

\minititle{ESEMPIO 4}
\qbox{Dopo un tiro sbagliato, B2, B4 e A10 si contendono il rimbalzo, mettendo entrambe le mani sulla palla in una situazione di palla trattenuta.\\a. La squadra A ha diritto alla rimessa per la freccia di possesso alternato.
\\b. La squadra B ha diritto alla rimessa per la freccia di possesso alternato.}{a. Rimbalzo offensivo di A4 (unico giocatore della squadra A coinvolto).
\\b. Rimbalzo difensivo di squadra B (più di un giocatore della squadra B è coinvolto).}

\minititle{ESEMPIO 5 (SITUAZIONI A RIMBALZO CON UN FALLO FISCHIATO O LA PALLA IN OOB)}
\qbox{a. Dopo un lay-up sbagliato del \#10 blue, il \#27 blu commette fallo prima che un giocatore abbia controllato la palla.}{2FGA \#10 blu - PF \#27 blu - fallo subito \#15 bianco - rimbalzo difensivo di squadra bianco.}

\qbox{b. Dopo un tiro in sospensione sbagliato del \#11 giallo, il \#4 blu tocca la palla mandandola fuori dal terreno di gioco.}{2FGA \#11 giallo - rimbalzo offensivo di squadra giallo.}

\minititle{ESEMPIO 6 (SITUAZIONI A RIMBALZO QUANDO SCADE IL PERIODO DI 24 SECONDI)}
\qbox{Dopo un 3FGA sbagliato del \#23 blu e prima che la palla sia controllata da un giocatore, il cronometro dei 24 secondi suona.}{3FGA \#23 blu - palla persa di squadra blu per SCV (shot clock violation) - nessun rimbalzo deve essere assegnato, perché la palla è morta.}

\qbox{Dopo un tiro in sospensione (\textcolor{violet}{da 3 punti}) sbagliato del \#13 verde, \#8 verde prende il rimbalzo ma non riesce a tirare prima che il cronometro dei 24 secondi suoni.}{3FGA \#13 verde - rimbalzo offensivo \#8 verde - palla persa di squadra verde per SCV (shot clock violation).}

\minititle{ESEMPIO 7}
\qbox{A2 sbaglia il primo di due tiri liberi.}{Nessun rimbalzo deve essere registrato, A2 ha diritto al secondo tiro libero e la palla è morta.}

\minititle{ESEMPIO 8}
\qbox{Il \#15 bianco sbaglia un FGA e prima che un giocatore conquisti il rimbalzo, il quarto termina.}{3FGA \#15 bianco - nessun rimbalzo deve essere registrato, la palla è morta.}

\minititle{ESEMPIO 9}
\qbox{Il \#32 bianco sbaglia un FGA, la palla non tocca l'anello, dopodiché rimbalza sul terrento e viene raccolta dal \#13 rosso.}{3FGA \#32 giallo - rimbalzo difensivo \#13 rosso.}

\minititle{ESEMPIO 10}
\qbox{Il \#21 giallo sbaglia un FGA. Diversi giocatori si contendono il rimbalzo, quando la palla è controllata dal \#10 giallo.}{FGA \#21 giallo - rimbalzo offensivo del \#10 giallo.}

\minititle{ESEMPIO 11}
\qbox{Il giocatore \#8 giallo sbaglia un FGA. Il \#10 bianco si contende il rimbalzo con altri giocatori ed è capace di deviare la palla in maniera controllata al \#8 bianco, che immediatamente palleggia a canestro realizza con una schiacciata.}{FGA \#8 bianco - rimbalzo offensivo \#10 bianco - 2FGM \#8 bianco - assist \#10 bianco.}

\minititle{ESEMPIO 12}
\qbox{Il \#20 giallo sbaglia un FGA. Il \#14 giallo salta per il rimbalzo, controlla la palla ma atterra fuori dal terreno di gioco.}{FGA \#20 giallo - rimbalzo difensivo di squadra blu.}

\minititle{ESEMPIO 13}
\qbox{Il \#13 rosso sbaglia un FGA. Il \#14 rosso sta lottando a rimbalzo, e negli ultimi istanti prima che la palla esca dal terreno di gioco, manda la palla contro la gamba del \#19 giallo, dopodiché la palla esce dal terreno di gioco. Alla squadra rossa viene assegnata una rimessa.}{FGA \#13 rosso - rimbalzo offensivo del \#14 rosso.}

\minititle{ESEMPIO 14}
\qbox{Dopo un fallo antisportivo, A3 sbaglia il secondo (\textcolor{violet}{ultimo!!!}) FTA.}{Nessun rimbalzo deve essere registrato.}
