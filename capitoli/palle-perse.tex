\section{PALLE PERSE}
\sectionline

\subsection{DEFINIZIONE}
\subsectionline

Una palla persa è un errore di un giocatore in attacco che consente alla squadra in difesa di acquisire il possesso della palla, incluso:

\begin{itemize}
    \item Un passaggio sbagliato
    \item Palleggio o palla scivolata
    \item Qualsiasi tipo di violazione o fallo offensivo
\end{itemize}

Solo la squadra in controllo di palla può commettere una palla persa. Una squadra è in controllo di palla quando:

\begin{itemize}
    \item Un giocatore sta tenendo o palleggiando la palla.
    \item La palla è a disposizione di un giocatore per una rimessa in gioco.
    \item La palla è a disposizione di un giocatore per un tiro libero.
    \item La palla è passata tra giocatori della stessa squadra.
\end{itemize}

Se la squadra in attacco si trova in una situazione di palla trattenuta, le statistiche da registrare dipenderanno dalla freccia di possesso alternato:

\begin{itemize}
    \item se la squadra in controllo della palla ha diritto alla rimessa, non deve essere registrata alcuna statistica.
    \item se la squadra in difesa ha diritto alla rimessa, deve essere registrata una palla persa al giocatore che ha commesso l'errore e una palla recuperata al giocatore in difesa coinvolto nella palla trattenuta.
\end{itemize}

\subsection{TIPI DI PALLE PERSE}
\subsectionline

\textbf{Gestione della palla}\\
Un giocatore in attacco perde il possesso mentre palleggia o maneggia la palla, o fallendo nel ricevere un passaggio che sarebbe stato possibile ricevere.

\textbf{Violazione}\\
Una violazione commessa da un giocatore in attacco o di squadra. I 5 secondi sulla rimessa, 8 secondi per superare la metà campo, violazione di 24 secondi, sono tutte \textbf{palle perse di squadra}. 

Tutte le altre violazioni (e.g., passi, doppio palleggio, 3 secondi in area, 5 secondi in possesso) sono palle perse individuali e devono essere registrate al giocatore.

\textbf{Fallo offensivo}\\
Un giocatore in attacco commette fallo. Qualsisasi fallo antisportivo o da espulsione commesso da un giocatore la cui squadra controlla la palla, è considerato una palla persa. Se commesso da un \underline{giocatore} (in quintetto) la palla persa verrà assegnata al giocatore, altrimenti verrà assegnata alla squadra.

\textbf{Passaggio}\\
Quando una squadra perde il possesso della palla a causa di un passaggio sbagliato, la palla persa deve essere registrata al giocatore che ha effettuato il passaggio, se lo statistico valuta che il passaggio fosse difficile da ricevere o non fosse possibile riceverlo. Altrimenti, la palla persa deve essere registrata al giocatore che avrebbe dovuto ricevere il passaggio.

\qbox{In alcune situazioni, una palla persa può essere classificata in diverse tipologie, ad esempio quando un passaggio sbagliato causa una violazione. Lo statistico deve riconoscere il modo in cui la palla persa ha avuto origine. Esempio: un giocatore sbaglia un passaggio, il suo compagno di squadra, per prenderlo esce dal terreno di gioco. In questo caso è stato il passaggio sbagliato a causare la violazione. Al passatore deve essere registrata una palla persa (passaggio).}{}

\subsection{MINI POSSESSI}
\subsectionline

Ci sono alcune situazioni in cui due o più palle perse avvengono quasi simultaneamente. Lo statistico deve decidere se la squadra ha acquisito il controllo della palla prima di perderla nuovamente. Per le palle perse, in presenza di dubbi sul controllo della palla, lo statistico deve assumere che non ci sia stato un controllo.

\subsection{ESEMPI}
\subsectionline

\minititle{ESEMPIO 1}
\qbox{Al \#6 giallo viene rubata la palla dal \#5 rosso mentre sta palleggiando.}{Palla persa (gestione della palla) \#6 giallo - palla rubata \#5 rosso. \textcolor{violet}{\sout{FGM \#10 rosso (punti contropiede) - assist \#7 rosso.}}}

\minititle{ESEMPIO 2}
\qbox{\#11 bianco passa la palla, che va direttamente fuori dal terreno di gioco.}{Palla persa (passaggio) \#11 bianco.}

\minititle{ESEMPIO 3}
\qbox{\#20 rosso fa un buon passaggio, ma il \#3 rosso non riesce a riceverlo e la palla esce dal terreno di gioco.}{Palla persa (gestione della palla) \#3 rosso.}

\minititle{ESEMPIO 4}
\qbox{\#22 giallo commette violazione di passi.}{Palla persa (violazione) \#22 giallo.}

\minititle{ESEMPIO 5}
\qbox{\#27 blu commette fallo bloccando illegalmente, mentre la sua squadra è in controllo di palla.}{PF \#27 blu - palla persa (fallo offensivo) \#27 blu. - fallo subito \#9 bianco.}

\minititle{ESEMPIO 6}
\qbox{La squadra bianca non riesce a tirare entro i 24 secondi.}{Palla persa di squadra bianca per SCV (shot clock violation).}

\minititle{ESEMPIO 7}
\qbox{A2 chiude il palleggio ed è marcato da vicino da B2. A2 non riesce a tirare o passare la palla entro 5 secondi.}{Palla persa (5 secondi) A2.}

\minititle{ESEMPIO 8}
\qbox{La squadra bianca è in possesso della palla quando \#11 blu e \#8 bianco commettono un fallo simultaneamente.}{Siccome le sanzioni sono compensate e la palla rimane alla squadra bianca, non deve essere registrata alcuna palla persa. Ai due giocatori verrà registrato un fallo tecnico.}

\minititle{ESEMPIO 9 (SITUAZIONE DI SALTO A DUE E FRECCIA DI POSSESSO ALTERNATO)}
\qbox{a. \#32 giallo sta palleggiando la palla che è toccata dal \#11 rosso, quando una palla trattenuta si verifica.}{La squadra gialla ha diritto alla rimessa, nessuna statistica verrà registrata.}

\qbox{b. Dopo che \#9 rosso riceve un passaggio, si verifica una situazione di palla trattenuta tra lui e il \#5 bianco. La squadra bianca ha diritto alla rimessa.}{Palla persa (gestione della palla) \#9 rosso - palla recuperata \#5 bianco.}

\minititle{ESEMPIO 10}
\qbox{Viene fischiato un fallo tecnico ad A2:
\\a. mentre A3 sta palleggiando.
\\b. subito dopo che B5 ha rubato la palla ad A3.}{a. T1 A2 - nessuna palla persa ad A2.
\\b. Palla persa (gestione palla) A3 - palla rubata B5 - T1 A2 - nessuna palla persa ad A2.}

\minititle{ESEMPIO 11}
\qbox{A4 è intrappolato nell'angolo da B5. Nel tentativo di salvare la situazione,
\\a. viene chiamata una violazione di passi contro A4;
\\b. A4 lancia un passaggio ad A1, che viene deviato da B5 e infine intercettato da B1;
\\c. A4 lancia un passaggio ad A1, che viene intercettato da B1 senza essere toccato da B5.}{a. Palla persa (passi) A4.
\\b. Palla persa (passaggio sbagliato) A4 - palla rubata B5.
\\c. Palla persa (passaggio sbagliato) A4 - palla rubata B1.}

\minititle{ESEMPIO 12}
\qbox{A3 tenta un passaggio ad A5, che viene deviato da B3. 
\\a. B5 sembra avere il controllo per una frazione di secondo prima di uscire dal campo;
\\b. B5 sembra avere il controllo per una frazione di secondo prima che A5 tolga di nuovo la palla a B5;
\\c. B5 raccoglie la palla persa, palleggia due volte e lancia un passaggio che viene intercettato da A4.}{a. Assumere nessun cambio controllo, quindi nessuna statistica deve essere registrata.
\\b. Assumere nessun cambio controllo, quindi nessuna statistica deve essere registrata.
\\c. Palla persa (passaggio sbagliato) A3 - palla rubata B3 - palla persa (passaggio sbagliato B5) - palla rubata A4.}

\minititle{ESEMPIO 13}
\qbox{\#5 bianco lancia un passaggio sbagliato a \#4 bianco, che commette violazione di RPZD.}{Palla persa (passaggio sbagliato) \#5 bianco.}
