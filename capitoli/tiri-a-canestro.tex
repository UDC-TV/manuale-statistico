\section{TIRI A CANESTRO}
\sectionline

\subsection{DEFINIZIONE}
\subsectionline

Un tentativo di tiro a canestro (FGA) è assegnato a un giocatore quando tira, lancia o palleggia il pallone verso il canestro avversario con l'intento di segnare.

Un tiro a canestro realizzato (FGM) è assegnato a un giocatore quando il suo tentativo di tiro a canestro entra nel canestro avversario e viene convalidato dall'arbitro, oppure viene convalidato a causa di una interferenza con il canestro o il tabellone da parte di un giocatore della squadra avversaria.

Un FGM conta sempre come FGA, che solitamente dovrebbe essere registrato automaticamente dal software.

\begin{itemize}
    \item Un FGA può avvenire in qualsiasi punto sul terreno di gioco, indipendentemente dalla meccanica di tiro.
    \item Un FGA a fine periodo conta come tale se la palla lascia le mani del tiratore prima del segnale acustico.
    \item A un giocatore che subisce fallo in atto di tiro non deve essere assegnato un FGA se il tiro non viene realizzato.
    \item Un FGA non deve essere assegnato se prima che la palla abbia lasciato le mani del tiratore, un giocatore della stessa squadra commette un fallo o una violazione. Quando il fallo o la violazione avvengono dopo che la palla ha lasciato le mani del tiratore, il FGA deve essere assegnato.
    \item Il \textit{tip} (anche noto come \textit{putback}) da parte di un giocatore in attacco conta come FGA (e come rimbalzo offensivo) se il giocatore è in controllo e tenta di realizzare. Se il \textit{tip} ha successo, conta come FGM, indipendentemente dal controllo della palla.
    \item Se un tiro è stoppato, un FGA è assegnato (indipendentemente dal fatto che la palla sia stata rilasciata o meno).
\end{itemize}

\subsection{AUTOCANESTRO}
\subsectionline

Se un giocatore della squadra in difesa segna accidentalmente nel proprio canestro, il canestro sarà assegnato al capitano in campo della squadra avversaria come FGM. Se questo canestro avviene a seguito di una situazione a rimbalzo dopo un FGA mancato dalla squadra in attacco, alla stessa verrà assegnato un rimbalzo offensivo. Nelle situazioni diverse da quelle a rimbalzo, se la stessa squadra è in possesso prima di segnare un autocanestro, una palla persa dovrà essere assegnata al giocatore che ha segnato l'autocanestro.

\subsection{INTERFERENZA}
\subsectionline

\begin{itemize}
    \item Nel caso di una interferenza a canestro da parte di un giocatore in difesa, un FGM deve essere assegnato al tiratore, anche se il tiro non fosse stato realizzato. Nessuna statistica deve essere assegnata al giocatore che ha commesso l'interferenza.
    \item Nel caso di una interferenza a canestro da parte di un giocatore in attacco, nessun FGA deve essere assegnato al tiratore. Una palla persa deve essere assegnata al giocatore che ha commesso l'interferenza.
\end{itemize}

\subsection{PUNTI CONTROPIEDE}
\subsectionline

\subsection{ESEMPI}
\subsectionline

\minititle{ESEMPIO 1}

\minititle{ESEMPIO 2}

\qbox{\#11 nero sbaglia un lay-up e il \#33 giallo effettua un \textit{tip} mandando accidentalmente la palla nel proprio canestro. Il \#9 nero è il capitano in campo della propria squadra durante questa azione.}{2FGA \#11 nero - rimbalzo offensivo di squadra nero - 2FGM \#9 nero.}

\minititle{ESEMPIO 3}
Tiro appena prima o appena dopo del segnale acustico di fine periodo.

\qbox{Con 4 secondi rimanenti nel quarto, \#24 blu ruba la palla e la passa al \#6 blu, che rilascia la palla (\textcolor{violet}{da metà campo}) verso il canestro avversario con 1 secondo residuo nel quarto.}{Palla persa (passaggio sbagliato) \#10 bianco - palla rubata \#24 blu - 3FGA \#6 blu.}

\minititle{ESEMPIO 4}
\qbox{\#8 verde tenta una schiacciata, che viene stoppata dal \#4 bianco, e la palla si appoggia sul supporto del canestro. La squadra in maglia bianca riceve la palla per la freccia di possesso alternato.}{2FGA \#8 verde - stoppata \#4 bianco - rimbalzo difensivo di squadra bianco.}

\minititle{ESEMPIO 5}
Gli esempi riportati di seguito riguardano situazioni in cui una squadra è in situazione di penalità e commette un fallo.

\qbox{La squadra B è in situazione di penalità. A1 sta andando a canestro quando subisce fallo da B5 prima del tiro.}{PF (fallo non su tiro) B5 - fallo subito A1 - nessun FGA registrato.}

\qbox{\#2 bianco sta andando a canestro e subisce fallo in atto di tiro dal \#7 blu.}{PF (fallo su tiro) \#7 blu - fallo subito \#2 bianco - nessun FGA registrato.}

\qbox{La squadra B è in situazione di penalità. A1 sta andando a canestro e subisce fallo da B5 dopo aver rilasciato la palla (\textcolor{violet}{???}).}{FGA A1 - rimbalzo offensivo di squadra A - fallo personale (non su tiro) B5 - fallo subito A1.}

\minititle{ESEMPIO 6}

\minititle{ESEMPIO 7}
\qbox{A1 sbaglia un FGA e B1 prende il rimbalzo. Immediatamente dopo il rimbalzo, B1 subisce fallo da A3 e ottiene 2 tiri liberi per il bonus. Realizza uno dei due tiri liberi.}{FGA A1 - rimbalzo difensivo di B1 - PF A3 - fallo subito B1 - 2(\textcolor{violet}{???}) FTM B1 (NON registrati come punti contropiede).}

\minititle{ESEMPIO 8}

