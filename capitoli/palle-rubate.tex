\section{PALLE RUBATE}
\sectionline

\subsection{DEFINIZIONE}
\subsectionline

Una rubata viene assegnata a un giocatore difensivo quando le sue azioni causano un palla persa a un giocatore avversario. Una rubata deve sempre comportare il contatto con la palla, ma non deve necessariamente essere controllata.

\begin{itemize}
    \item Intercetto o deviazione di un passaggio
    \item Togliere la palla a un avversario che la trattiene o la palleggia.
    \item Raccogliere una palla vagante dopo un errore di un giocatore in attacco.
\end{itemize}

Non viene accreditata alcuna rubata se la palla diventa morta e alla squadra in difesa viene assegnato il possesso della palla fuori dal campo, anche se il palla persa è stato causato dalle azioni del giocatore in difesa.

L'unico caso in cui una rubata può essere accreditata quando la palla diventa morta è quando le azioni di un giocatore difensivo causano una palla trattenuta e la squadra di questo giocatore vince il possesso come risultato della regola del possesso alternato.

Se una rubata viene accreditata a un giocatore difensivo, deve esserci un palla persa corrispondente accreditata a un giocatore offensivo. (Non vale il contrario: il verificarsi di un palla persa non significa sempre che si sia verificata una rubata).

In tutte le situazioni che coinvolgono più di un giocatore difensivo, la rubata viene accreditata al giocatore che per primo ha deviato la palla e avviato il palla persa.

\subsection{ESEMPI}
\subsectionline

\minititle{ESEMPIO 1}
\qbox{\#10bianco sta palleggiando quando sbaglia la palla, che rimbalza verso \#9 verde, che la recupera senza muoversi.}{Palla persa (gestione della palla) \#10 bianco - rubata \#9 verde}

\minititle{ESEMPIO 2}
\qbox{Il \#12 rosso sta palleggiando la palla quando il \#15 giallo la sottrae al \#7 giallo.}{Palla persa (gestione della palla) \#12 rosso - rubata \#15 giallo.}

\minititle{ESEMPIO 3}
\qbox{Il \#11 bianco effettua una giocata difensiva per deviare un passaggio del \#30 blu. La palla tocca nuovamente \#30 blu prima di uscire dal campo.}{Palla persa (fuori dal campo) \#30 blu ma nessuna rubata per \#11 bianco, poiché la palla diventa morta.}

\minititle{ESEMPIO 4}
\qbox{Il \#15 giallo devia un passaggio di \#21 rosso destinato a \#14 rosso, che in un'azione di riflesso cerca di prendere la palla, ma può solo deviarla oltre la linea laterale. La squadra B ha diritto a una rimessa in gioco dalla linea laterale.}{Palla persa (passaggio) A4 ma senza rubata.}

\minititle{ESEMPIO 5}
\qbox{\#11 bianco dribbla la palla quando questa viene fatta cadere dal \#24 nero. Il \#23 bianco e il \#45 nero afferrano entrambi il pallone perso e viene chiamata una palla trattenuta. Al nero viene assegnata la palla per una rimessa dalla linea laterale (regola del possesso alternato).}{Palla persa (gestione della palla) \#11 bianco - Furto \#24 nero. Non vengono accvioletitate statistiche a \#23 bianco e \#45 nero.}

\minititle{ESEMPIO 6}
\qbox{Il \#11 nero sta guidando verso il canestro. Il difensore \#12 giallo anticipa il gioco e commette un fallo offensivo contro \#11 nero.}{Palla persa (fallo offensivo) di \#11 nero. Non viene registrata alcuna rubata.}
