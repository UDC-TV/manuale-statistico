\section{TIRI LIBERI}
\sectionline

\subsection{DEFINIZIONE}
\subsectionline

Un tentativo di tiro libero (FTA) è assegnato a un giocatore quando tira un tiro libero, a meno che non avvenga una violazione da parte di un giocatore in difesa e il tiro è sbagliato. Al giocatore non deve essere assegnato un FTA che è influenzato dalle azioni illegali di un avversario, a meno che il tiro non venga realizzato (FTM).

Un FTM viene assegnato a un giocatore quando il suo tentativo di tiro libero entra nel canestro avversario e viene convalidato dall'arbitro.

Un FTM conta sempre come FTA, che solitamente dovrebbe essere registrato automaticamente dal software.

In caso di violazione durante i tiri liberi, gli statistici devono osservare in maniera attenta cosa stanno segnalando gli arbitri, a chi viene fischiata la violazione e qual è il risultato della chiamata. Si applicano le seguenti statistiche:

\begin{itemize}

    \item Quando \textbf{un giocatore della squadra in difesa} commette una violazione:
        \begin{itemize}
            \item Se il tiro libero è realizzato, il punteggio conterà nonostante la violazione difensiva, e il giocatore che ha realizzato il tiro libero riceverà un FTM e un FTA.
            \item Se il tiro libero è sbagliato, il tiro libero sarà ripetuto. Il giocatore che ha sbagliato il tiro libero riceverà un FTA, ed eventualmente un FTM. Il tiro libero annullato e sbagliato non deve essere registrato come FTA ma semplicemente ignorato.
        \end{itemize}

    \item Quando \textbf{il tiratore} commette una violazione:
        \begin{itemize}
            \item Se il tiro libero è realizzato, verrà cancellato.
            \item Un FTA sarà assegnato al tiratore.
            \item Se il tiro è l'ultimo di una serie di tiri liberi, alla squadra in difesa verrà assegnata la palla per la rimessa in gioco. Registrare la squadra in difesa un rimbalzo di squadra.
        \end{itemize}

    \item Quando \textbf{un giocatore della squadra in attacco, diverso dal tiratore} commette una violazione:
        \begin{itemize}
            \item Gli arbitri non annulleranno il tiro libero, quindi dovrà essere assegnato un FTG al tiratore.
            \item Se il tiro libero è sbagliato, al tiratore sarà comunque assegnato un FTA. Se il tiro è l'ultimo di una serie di tiri liberi, alla squadra in difesa verrà assegnata la palla per la rimessa in gioco. Registrare la squadra in difesa un rimbalzo di squadra.
        \end{itemize}

\end{itemize}

In tutti i casi sopra riportati, nessuna palla persa ha avuto luogo.

Se un giocatore sbagliato sta tentando un tiro libero, gli arbitri cancelleranno i FTM e FTA risultanti dall'errore. Questo verrà registrato come palla persa di squadra.

Se vengono assegnati tiri liberi all'inizio del quarto, come risultato di un fallo tecnico, FTA ed eventuale FTM devono essere registrati nel periodo.

In tutte queste situazioni speciali, è importante che gli statistici osservino attentamente le azioni degli arbitri e registrino le statistiche in modo accurato, comunicando quando necessario con gli ufficiali di campo.

\subsection{ESEMPI}
\subsectionline

\qbox{A1 tira un tiro libero, durante il quale B3 commette violazione:
\\a. il tiro è realizzato.
\\b. Il tiro è sbagliato.}
{a. FTM A1.
\\b. tiro libero annullato e dovrà essere ripetuto $\Rightarrow$ nessun FTA deve essere registrato.}

\qbox{A5 tira un ultimo tiro libero, durante il quale A4 commette una violazione:
\\a. il tiro è realizzato.
\\b. Il tiro è sbagliato.}
{a. FTM A5.
\\b. FTA A5 - rimbalzo difensivo di squadra B. Nessuna palla persa deve essere registrata.}
