\section{ASSIST}\sectionline
\subsection{DEFINIZIONE}
\subsectionline

Un assist è un passaggio che porta direttamente al canestro realizzato di un compagno di squadra.

\begin{itemize}
    \item Un passaggio  a un giocatore all'interno del pitturato (area dei 3 secondi), che segna dall'interno del pitturato, è sempre considerato un assist. Per "dentro il pitturato" si intende la presenza di uno o entrambi i piedi all'interno del pitturato o sulla linea, e il giocatore deve segnare senza uscire dal pitturato (entrambi i piedi fuori dal pitturato) prima di segnare.
    \item Un passaggio a un giocatore fuori dal pitturato, che segna senza palleggiare, è sempre un assist.
    \item Un passaggio a un giocatore fuori dal pitturato, che segna dopo uno o più palleggi, è considerato un assist se il tiratore non deve battere il suo  difensore. Non viene assegnato un assist se il tiratore batte il suo difensore in una situazione di 1 contro 1 e il difensore è rivolto verso il tiratore e si  trova direttamente di fronte a lui, tra il tiratore e il canestro. I difensori che contestano un tiro non sono rilevanti in questo senso. Un assist viene comunque assegnato quando l'attaccante sta guidando oltre il difensore in una situazione di 1 contro 1 se:
    \begin{itemize}
        \item si dirige verso il canestro immediatamente dopo aver ricevuto il passaggio \textbf{E}
        \item il difensore viene colto fuori equilibrio.
    \end{itemize}
\end{itemize}

Lo stesso principio si applica alle situazioni di contropiede, con un passaggio a un giocatore a metà campo.

Se il giocatore che riceve il passaggio subisce un fallo nell'atto di tirare e realizza almeno un tiro libero, viene assegnato un assist come per un FGM.

Inoltre, si applicano sempre le seguenti regole generali:

\begin{itemize}
    \item Ogni volta che un giocatore segna può essere assegnato un solo assist.
    \item Solo l'ultimo passaggio prima di un tiro può essere un assist (anche se il penultimo passaggio ha creato l'azione).
    \item La distanza, il tipo di tiro e la facilità con cui un giocatore segna non sono rilevanti.
    \item In una situazione di contropiede non si può assegnare alcun assist se il giocatore riceve il passaggio nella metà campo della squadra prima di dirigersi verso il canestro (coast to coast).
    \item L'assist non viene assegnato se il passaggio viene chiaramente deviato e finisce a un giocatore diverso da quello a cui era inizialmente destinato.
\end{itemize}

\subsection{ESEMPI}
\subsectionline

\minititle{ESEMPIO 1}
\qbox{Dopo che \#31 verde cattura un rimbalzo difensivo, il \#2 verde effettua un passaggio a \#8 verde, che sbaglia un lay-up ma ha abbastanza tempo per segnare dal rimbalzo.}{No assist: c'è stato un FGA e un rimbalzo offensivo tra il passaggio e il tiro realizzato. 2FGA \#8 verde - rimbalzo offensivo \#8 verde - 2FGM \#8 verde (punti in contropiede).}

\minititle{ESEMPIO 2}
\qbox{Il \#7 rosso passa al \#9 rosso, che esita e poi prende e realizza il tiro.}{Il \#9 rosso segna senza dribblare, quindi si tratta di un assist per il \#7 rosso. 3FGM \#9 rosso - assist \#7 rosso.}

\minititle{ESEMPIO 3}
\qbox{Il \#11 bianco passa al \#5 bianco, che effettua un dribbling per trovare l'equilibrio e poi tira, \textcolor{violet}{segnando}.}{2FGM \#5 - assist \#11 bianco.}

\minititle{ESEMPIO 4}
\qbox{Il giocatore \#7 rosso effettua un passaggio a tutto campo al giocatore \#9 rosso, che deve solo passare la palla al giocatore \#10 rosso per un lay-up che viene realizzato.}{Anche se il primo passaggio di \#7 rosso ha creato l'opportunità, non è stato l'ultimo passaggio prima del punteggio. Attribuire l'assist a \#9 rosso.}

\minititle{ESEMPIO 5 (SITUAZIONI CON AZIONE SIGNIFICATIVA DEL TIRATORE PRIMA DEL TIRO)}
\qbox{a. Il \#21 giallo passa la palla al \#11 giallo all'interno del pitturato. \#11 giallo, pur se marcato da vicino, supera il difensore e segna con un lay-up.}{2FGM \#11 giallo - assist \#21 giallo.}

\qbox{b. \#7 blu passa la palla a \#3 blu, che è sorvegliato da vicino e riceve il passaggio fuori dal pitturato, per una schiacciata al centro.}{2FGM \#3 blu - nessun assist.}

\minititle{ESEMPIO 6}
\qbox{Il \#10 bianco ruba la palla, il \#4 bianco effettua un passaggio al \#10 bianco in zona di attacco, che segna con un lay-up.}{2FGM \#10 bianco - assist \#4 bianco.}

\minititle{ESEMPIO 7}
\qbox{Dopo un tiro sbagliato, \#10 blu ottiene il rimbalzo difensivo. Passa la palla a \#5 blu nel backcourt, che va coast to coast e segna con un lay-up.}{Nessun assist per il \#10 blu, poiché il \#5 ha ricevuto il passaggio nella propria metà campo.}

\minititle{ESEMPIO 8}
\qbox{Dopo un tiro sbagliato, A2 ottiene il rimbalzo difensivo. Fa un passaggio lungo ad A3 a metà campo, davanti alla difesa.
\\a. B3 commette un fallo su A3 per fermare il contropiede e le viene fischiato un fallo antisportivo. 
\\b. B4 commette un fallo personale (di tiro) mentre A3 tenta un lay-up. In entrambe le situazioni, ad A3 vengono assegnati due tiri liberi, realizzando il primo e sbagliando il secondo.}
{a. Fallo antisportivo B3 - Fallo di tiro A3 - FTM A3 (punti in contropiede, nessun assist).
\\b. Fallo personale B4 - Fallo di gioco A3 - FTM A3 (punti in contropiede) - assist A2.}

\minititle{ESEMPIO 9}
\qbox{Il \#12 giallo effettua un inbound sulla linea di fondo sotto il canestro dell'avversario e passa la palla al \#21, che segna un FGM.}{3FGM \#21 giallo - assist \#12 giallo.}
